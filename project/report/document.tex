\documentclass{sig-alternate-05-2015}

\begin{document}

\title{Deep Learning Movie Recommendations}
\author{Jeffrey Lund}
\date{}
\maketitle

\begin{abstract}
This is an abstract.
\end{abstract}

\section{Introduction}

Increasingly streaming movie services like Netflix, Hulu, Amazon Prime, and
others are how consumers enjoy video content.
For example, in 2017 Netflix subscribers collectively watch more than 140
million hours per day.
For streaming movie services like Netflix, recommendation systems are
important for helping users to discover new content to enjoy.
In fact, roughly 80\% of hours streamed at Netflix were influenced by their
proprietary recommendation system~\cite{netflix}.
Considering that Netflix is on track to exceed \$11 billion in revenue this
year, the importance of movie recommendation systems cannot be understated---
they are an integral part of how we consume video content today.
With this in mind, the problem I will examine is movie recommendations through
collaborative filtering.

In the context of movie recommendation, collaborative filtering aims to predict
unknown movie ratings for a particular user based on that user's known ratings
as well as the movie ratings by other users in the system.
The underlying assumption is that if we can accurately predict movie ratings,
then we can recommend new movies to users by recommending movies they are
likely to highly rate, including films the user may not have considered before.

The most common method of performing collaborative filtering is to utilize a
k-nearest-neighbor approach with correlation similarity measure to users which
are similar to one another~\cite{user-user}.
This relies on the assumption that two users rated the same item similarly,
they are likely to rate other items similarly as well.
At scale, data structures such as ball trees~\cite{ball-tree} and k-d trees (a
binary space partition tree in k-dimensions) have been utilized to more
efficiently compute local neighbors between user profiles.

An alternative k-nearest-neighbor approach instead computes similarity between
pairs of items (as opposed to users) with the idea that users who like a
particular item will like similar items~\cite{item-item}.
Since there tends to be many more users than items in a recommendation system,
user-user collaborative filtering can be much performant, although in my
preliminary experiments with movie ratings indicate that user-user is more
accurate.

Another common method for performing collaborative filtering is with matrix
factorization~\cite{matrix-factorization}.
With this technique a user-item matrix is factorized into two matrices with the
inner dimension representing some latent factors.
The resulting factorization represents both users and items in terms of the
latent factors in such a way that new items can be recommended to users based
on the latent factors.
As with item-item neighborhood approaches, my preliminary experiments on movie
ratings indicate that user-user neighborhood approaches are superior to matrix
factorization.

Deep learning has revolutionized many fields of computer science, including
natural language processing~\cite{deep-survey}.
Despite this, deep learning is relatively new in the area of recommendation
systems, and has not received much attention~\cite{dl-recsys-survey}.
The aim of this project is to experiment with deep learning for collaborative
filtering on a large set of movie ratings.

\section{Model Architecture}

\section{Experimental Setup}

\section{Results}

\section{Conclusion}

\bibliographystyle{abbrv}
\bibliography{report}

\end{document}
