\documentclass[12pt]{article}
\usepackage{url}

\title{}
\date{}

\begin{document}

\maketitle

%\section*{Abstract}

\section{Introduction}

For streaming movie services like Netflix, recommendation systems are hugely
important for helping users to discover new content to enjoy.
In fact, roughly 80\% of hours streamed at Netflix were influenced by their
proprietary movie recommendation system~\cite{netflix}.
With this in mind, the problem I propose to work on is movie recommendations
through collaborative filtering.

Collaborative filtering is an approach which recommendation systems which
relies on the ratings and tags for particular user as well as the ratings and
tags of similar users.
As opposed to content-based systems, collaborative filtering accounts for
users with diverse taste, so long as there are other users with similar
preferences.
There are many ways to perform collaborative filtering such as utilizing
k-nearest neighbor clustering with user profiles~\cite{user-user}, or
item-item systems which compute similarity between items with the idea that
users who like a particular item will like similar items~\cite{item-item}.

For my project, I propose working on a collaborative filtering problem for
movie recommendations.
I will work with the latest MovieLens dataset~\cite{movielens},
which contains 26,000,000 ratings and 750,000 tags applied to 45,000 movies by
270,000 users.
Using the ratings and tags provided by users, I will try to predict user
ratings of movies in order to produce novel movie recommendations.

\section{Method}

\section{Conclusions}

\bibliographystyle{annotate}
\bibliography{infobib}

\end{document}
