\documentclass[12pt]{article}
\usepackage{url}
\usepackage{fullpage}
\usepackage{titlesec}
\titlespacing\section{0pt}{12pt plus 4pt minus 2pt}{0pt plus 2pt minus 2pt}

\title{Deep Learning to Predict Movie Ratings}
\author{Jeffrey Lund}
\date{}

\begin{document}

\maketitle

\section*{Abstract}

For services like Netflix, recommendation systems are essential for helping
users find new movies to enjoy.
Using the MovieLens dataset, I propose using deep learning to predict users'
ratings on new movies.
This approach also allows us to multiply the amount of training data by
withholding each individual rating one at a time.
Using these predicted ratings, I can recommend new movies to users.

\clearpage

\section{Introduction}

For streaming movie services like Netflix, recommendation systems are
important for helping users to discover new content to enjoy.
In fact, roughly 80\% of hours streamed at Netflix were influenced by their
proprietary recommendation system~\cite{netflix}.
With this in mind, the problem I propose to work on is movie recommendations
through collaborative filtering.

Collaborative filtering is an approach which recommendation systems which
relies on the ratings for particular user as well as the ratings
of similar users.
As opposed to content-based systems, collaborative filtering accounts for
users with diverse taste, so long as there are other users with similar
preferences.
There are many ways to perform collaborative filtering such as utilizing
k-nearest neighbor clustering with user profiles~\cite{user-user}, or
item-item systems which compute similarity between items with the idea that
users who like a particular item will like similar items~\cite{item-item}.

For my project, I propose working on a collaborative filtering problem for
movie recommendations.
I will work with the latest MovieLens dataset~\cite{movielens},
which contains 26,000,000 ratings applied to 45,000 movies by
270,000 users.
My goal will be to use this large movie rating database to produce novel movie
recommendations.

\section{Method}

Rather than computing similarity scores between users or items to find the
nearest neighbors, I will employ deep learning to predict user ratings on new
movies.
By predicting the user rating of new movies, I can show users previously
unseen movies that they are likely to enjoy.
While the exact structure of the deep neural network I will use is subject to
change, I plan to use a vector-space representation of a user profile as input
along with a one-hot encoding of the movie I wish to rate.

One advantage of this approach is that I can drastically multiply the amount
of training data I have by simply withhold a single rating from a particular
user profile.
In other words, if a user has rated $n$ different movies, I can produce $n$
different training examples by withholding one of the $n$ ratings and training
the network to predict the missing rating.

Once I have trained the network to predict movie ratings given a user profile,
producing recommendations is easy: simply select the movies with have the
highest predicted ratings. I can then evaluate the overall system using
standard metrics for recommendation systems such as prediction accuracy or
rank accuracy.

\section{Conclusions}

Deep learning has revolutionized many fields of computer science, including
natural language processing~\cite{deep-survey}.
It appears that deep learning is starting to take hold in information
retrieval as well, including recommendation systems~\cite{deep-cf}.
However, to my knowledge it seems like deep learning is still relatively new
in this field, so the best approaches may not have been discovered yet.
This proposal suggest one possible way of employing deep learning to solve an
important problem in information retrieval.

\bibliographystyle{annotate}
\bibliography{infobib}

\end{document}
