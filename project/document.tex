\documentclass[12pt]{article}
\usepackage{url}

\title{Reranking CQA Query Results with Deep Learning}
\date{}

\begin{document}

\maketitle

\section*{Abstract}

This is an abstract. It summarizes the stuff I say below.

\section{Introduction}

In contrast to web search, which aims to retrieve relevant documents in
response to a query, community question answering (hereafter CQA) systems such
as Yahoo! Answers\footnote{\url{http://answers.yahoo.com}} or
Naver\footnote{\url{http://www.naver.com}} seek to give users direct answers
to questions written with natural language.
However, as these systems acquire more and more answers, the ability to match
new questions to previous question and answer pairs becomes increasingly
important in order to avoid duplication of work.

One approach is the one we covered in class~\cite{rightfact}.
For factoid type queries, we can just do a simple machine learning thing.
They use features and stuff.

\section{Method}

With respect to my chosen problem,
I'm gonna use a Siamese style network~\cite{siamese} trained with contrastive
loss.
Except I'm gonna take an approach similar to skip-gram where I try and get
close things to be close and far things to far.

\section{Conclusions}

Remark on uniqueness and merit of solution.
Deep learning is revolutionary in NLP.
Not much has been done in IR, but perhaps it is useful.
If this works out, it might be evidence that perhaps deep learning has been
overlooked in IR.


\bibliographystyle{annotate}
\bibliography{infobib}

\end{document}
