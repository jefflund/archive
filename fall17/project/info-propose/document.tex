\documentclass[12pt]{article}
\usepackage{url}
\usepackage{fullpage}
\usepackage{titlesec}
\titlespacing\section{0pt}{12pt plus 4pt minus 2pt}{0pt plus 2pt minus 2pt}

\title{Movie Recommendations with Deep Learning}
\author{Jeffrey Lund}
\date{}

\begin{document}

\maketitle

\section*{Abstract}

For streaming movie services like Netflix, recommendation systems are
essential for helping users find new movies to enjoy. Using the MovieLens
dataset, I propose to explore the use of deep learning to predict users'
ratings on new movies, thereby enabling movie recommendations. I will compare
this approach to standard collaborative techniques such as k-nearest-neighbor
and matrix-factorization.

\section{Introduction}

For streaming movie services like Netflix, recommendation systems are
important for helping users to discover new content to enjoy.
In fact, roughly 80\% of hours streamed at Netflix were influenced by their
proprietary recommendation system~\cite{netflix}.
Considering that Netflix is on track to exceed \$11 billion in revenue this
year, the importance of movie recommendation systems cannot be understated---
they are an integral part of how we consume video content today.
With this in mind, the problem I propose to work on is movie recommendations
through collaborative filtering.

Collaborative filtering is an approach for recommendation systems which
relies on the ratings for particular user as well as the ratings
of similar users.
As opposed to content-based systems, collaborative filtering accounts for
users with diverse taste, so long as there are other users with similar
preferences.
By finding similar users, new items can be recommended based on the assumption
that items which are liked by similar users will be liked by the user in
question.

There are many ways to perform collaborative filtering such as utilizing
k-nearest neighbor clustering with user profiles~\cite{user-user}.
Various approaches for measuring similarity have been proposed, but a simple
approach is to represent a user profile as a vector, and then use some measure
of similarity between those vectors (e.g., cosine similarity).
An alternative k-nearest-neighbor approach instead computes similarity between
pairs of items with the idea that users who like a particular item will like
similar items~\cite{item-item}.

Another common method for performing collaborative filtering is with matrix
factorization~\cite{matrix-factorization}.
With this technique a user-item matrix is factorized into two matrices with the
inner dimension representing some latent factors.
The resulting factorization represents both users and items in terms of the
latent factors in such a way that new items can be recommended to users based
on the latent factors.

With movie recommendation in mind, for my project, I will explore deep learning
as alternative to nearest-neighbor or matrix factorization for collaborative
filtering.
I will work with the latest version of the MovieLens dataset~\cite{movielens}, 
which is is the version recommended for education and development.
This dataset contains 26,000,000 ratings applied to 45,000 movies by 270,000
users.
My goal will to use the large movie ratings database to produce a deep neural
network which can accurately predict user ratings on new movies.
I will compare this approach to existing nearest neighbor and matrix
factorization techniques.

\section{Method}

Deep learning has revolutionized many fields of computer science, including
natural language processing~\cite{deep-survey}.
Deep learning, which is essentially just deep artificial neural networks, is
able to learn complex decision boundaries for classification, or complex
non-linear regressions.
By stacking large numbers of hidden layers in these networks, deep neural
networks are able to learn complex functions by learning to extract many low
level features from the data and composing them in useful non-linear
combinations.

\subsection{Network Architecture}

While neural networks are theoretically able to approximate any computable
function, including the mapping from user profiles to movie ratings, in
practice great care must be taken when selecting the architecture of the
neural network. While the extract structure of my network is subject to change,
there are some reasonable starting places.

\textbf{Inputs} The input to my network architecture will be two
$n$-dimensional vectors, where $n$ is the number of movies in the MovieLens
database.
One vector will encode a particular user profile, with each dimension
indicating the rating the user gave for a particular film (or a zero to
indicate that no rating has been given).
The other vector will be a one-hot encoding (i.e., a vector with a single
``hot'' dimension set to 1, with all other values set to zero) of particular
item.
These two vectors will request that the network predict a rating for a
particular user for a particular movie.

One advantage of this input format is that I can without a single rating from
a known user profile, and use the known rating for withheld item as a labeled
examples. Consequently, even though I only have 270,000 user profiles, each
one of the 26,000,000 individual ratings constitutes a train example.

\textbf{Hidden Layers} There are a variety of ways to structure a simple
feed-forward neural network.  I will start with a number of the standard
fully-connected layers.  However, I will also experiment with alternative
structures, such as ResNets~\cite{resnets}, which currently obtain
state-of-the-art results in other fields such as image recognition.

\textbf{Output} For this project, there are two main possibilities for the
output of my network.
The first is to treat this problem as a classification problem, with five
different class representing the five start ratings that are present in the
data.
Under this architecture, I would likely treat the five outputs of my network
as unnormalized log probabilities, and use cross entropy as my loss function.

Alternatively, I may treat this problem as a regression problem, and have my
network learn a non-linear regression between user profiles and predicted
rating.
In this case, I would likely use root mean square error as my loss function.

It is unclear which approach will yield the best results, so I will try both
classification and regression with various network architectures.
However, once I have predict ratings, making movie recommendations is simple:
simply recommend the movies with the highest predicted ratings.
Assuming I am better able to predict user ratings on new movies, I should be
able to improve the movie recommendations.

\subsection{Evaluation}

In order to evaluate the performance of my system, I will withhold a test set
of ratings from the training data.
Each rating is accompanied by a timestamp, so I will be able to split the data
based on when ratings were made such that older ratings are used as training
data, and newer ratings are used as test.
In this way, I'll be simulating what a streaming movie service like Netflix
might actually observe.

I will evaluate the prediction performance using several metrics including
accuracy, root mean squared error, and mean absolute error.
Additionally, I can measure precision-at-K to see if the movies users actually
watched are recommended by my system.
For baselines, I will use both k-nearest-neighbor and matrix-factorization
approaches for collaborative filtering.

\section{Conclusions}

Despite the successes of deep learning in other fields~\cite{deep-survey},
it appears that deep learning is starting to take hold in information
retrieval as well, including recommendation systems~\cite{deep-cf}.
However, to my knowledge it seems like deep learning is still relatively new
in this field, so the best approaches may not have been discovered yet.
This proposal suggest one possible way of employing deep learning to solve the
problem of collaborative filtering.
If successful, I will have demonstrated that deep learning can outperform the
most common approaches for solving this problem.

\bibliographystyle{annotate}
\bibliography{infobib}

\end{document}
